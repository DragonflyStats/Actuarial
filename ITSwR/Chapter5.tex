ITSwR Section 5
 
5.1 Purpose
 
Trends in time series can be classified as either stochastic or deterministic. We may consider a trend to be stochastic when it shows inexplicable changes in direction, and we attributed apparent transient trends to high serial correlation with random error.
 
Deterministic trends and seasonal variation can be modelled using regression.
 
The logarithmic transformation, which is often used to stabilise the variance, is also considered.
 
5.6 Harmonic seasonal models
 
5.6.1 Simulation
5.6.2 Fit to simulated series 
 
5.7 Logarithmic transformations
 
xt=mtstzt
 

 
5.8 Non linear models
 
A linear model (e.g. a straight line trend) could be fitted to produce for {xt} the model
 
 
 
0 and 1 are model parameters and {zt} is a residual time series that may be autocorrelated.
 
 
5.9 Forecasting from regression
 
5.9.1. Introduction
 
A prediction is a forecast into the future. A forecast involves extrapolating a fitted model into the future by evaluating the model function for a new series of times.
 
5.9.2 Prediction in R
 
5.10 Inverse transform and bias correction
 
5.10.1 log-normal residual errors.
 
5.10.2 Empirical correction factor for forecasting means.
 
The correction factor can be used when the resiudal series of fitted log-regression model is Gaussian white noise.
