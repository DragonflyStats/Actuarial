
%===================================================%
\begin{frame}
\frametitle{Time Series Analysis}
\textbf{PART G CYCLICAL FORECASTING AND BUSINESS INDICATORS}
\begin{itemize}
\item
As indicated in Section Part F, forecasting based on the trend and seasonal components of a time series is
considered only a beginning point in economic forecasting.
\item  One reason is the necessity to consider the likely
effect of the cyclical component during the forecast period, while the second is the importance of identifying the
specific causative factors that have influenced the time series variables.
\end{itemize}
\end{frame}
%===================================================%
\begin{frame}
\frametitle{Time Series Analysis}
\begin{itemize}
\item For short-term forecasting, the effect of the cyclical component is often assumed to be the same as included
in recent time series values.
\item However, for longer periods, or even for short periods during economic instability,
the identification of the cyclical turning points for the national economy is important. 
\item Of course, the cyclical
variations associated with a particular product may or may not coincide with the general business cycle.
\end{itemize}
\end{frame}
%===================================================%
\begin{frame}
\frametitle{Time Series Analysis}
\textbf{EXAMPLE 4.}
\begin{itemize} 
\item Historically, factory sales of passenger cars have coincided closely with the general business cycle for the
national economy.
\item On the other hand, sales of automobile repair parts have often been countercyclic with respect to the
general business cycle.
\end{itemize}
\end{frame}
%===================================================%
\begin{frame}
\frametitle{Time Series Analysis}
\textbf{Leading Indicators}
\begin{itemize}
\item The National Bureau of Economic Research (NBER) has identified a number of published time series that
historically have been indicators of cyclic revivals and recessions with respect to the general business cycle.
\item One group, called leading indicators, has usually reached cyclical turning points prior to the corresponding
change in general economic activity. 
\item The leading indicators include such measures as average weekly hours
worked in manufacturing, value of new orders for consumer goods and materials, and a common stock price
index.
\end{itemize}
\end{frame}
%===================================================%
\begin{frame}
\frametitle{Time Series Analysis}
\textbf{ Coincident Indicators}\\
\begin{itemize}
\item A second group, called coincident indicators, are time series which have generally had turning points
coinciding with the general business cycle.
\item Coincident indicators include such measures as the employment rate
and the index of industrial production. 
\end{itemize}
\end{frame}
%===================================================%
\begin{frame}
\frametitle{Time Series Analysis}
\textbf{Lagging Indicators}\\
\begin{itemize}
\item The third group, called lagging indicators, are those time series for
which the peaks and troughs usually lag behind those of the general business cycle. 
\item Lagging indicators include
such measures as manufacturing and trade inventories and the average prime rate charged by banks.
\end{itemize}
\end{frame}
%===================================================%
\begin{frame}
\frametitle{Time Series Analysis}
In addition to considering the effect of cyclical fluctuations and forecasting such fluctuations, specific
causative variables that have influenced the time series values historically should also be studied. Regression
and correlation analysis (see Chapters 14 and 15) are particularly applicable for such studies as the relationship
between pricing strategy and sales volume. Beyond the historical analyses, the possible implications of new
products and of changes in the marketing environment are also areas of required attention.
\end{frame}
\end{document}
%===================================================%
\begin{frame}
\frametitle{Time Series Analysis}
16.8 FORECASTING BASED ON MOVING AVERAGES
A moving average is the average of the most recent n data values in a time series. This procedure can be
represented by
MA ¼
S (most recent n values)
n
(16:16)
As each new data value becomes available in a time series, the newest observation replaces the oldest
observation in the set of n values as the basis for determining the new average, and this is why it is called a
moving average.
\end{frame}
%===================================================%
\begin{frame}
\frametitle{Time Series Analysis}
\begin{itemize}
\item The moving average can be used to forecast the data value for the next (forthcoming) period in the time
series, but not for periods that are farther in the future. 
\item It is an appropriate method of forecasting when there is
no trend, cyclical or seasonal influence on the data, which of course is an unlikely situation. 
\item The procedure
serves simply to average out the irregular component in the recent time series data. See Problem 16.8.
\end{itemize}
\end{frame}
