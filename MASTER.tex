\section*{Time Series and Forecasting}
R has extensive facilities for analyzing time series data. This section describes the creation of a time series, seasonal decompostion, modeling with exponential and ARIMA models, and forecasting with the forecast package.
%================================================================================= %
\subsection{Creating a time series}
The ts() function will convert a numeric vector into an R time series object. The format is ts(vector, start=, end=, frequency=) where start and end are the times of the first and last observation and frequency is the number of observations per unit time (1=annual, 4=quartly, 12=monthly, etc.).

\begin{framed}
	\begin{verbatim}
# save a numeric vector containing 48 monthly observations
# from Jan 2009 to Dec 2014 as a time series object
myts <- ts(myvector, start=c(2009, 1), end=c(2014, 12), frequency=12) 

# subset the time series (June 2014 to December 2014)
myts2 <- window(myts, start=c(2014, 6), end=c(2014, 12)) 

# plot series
plot(myts)
\end{verbatim}
\end{framed}
%================================================================================= %
\subsection*{Seasonal Decomposition}
A time series with additive trend, seasonal, and irregular components can be decomposed using the \texttt{stl()} function. Note that a series with multiplicative effects can often by transformed into series with additive effects through a log transformation (i.e., newts <- log(myts)).
# Seasonal decomposition
fit <- stl(myts, s.window="period")
plot(fit)

\begin{framed}
\begin{verbatim}
# additional plots
monthplot(myts)
library(forecast)
seasonplot(myts)
\end{verbatim}
\end{framed}
Exponential Models
Both the HoltWinters() function in the base installation, and the ets() function in the forecast package, can be used to fit exponential models.
# simple exponential - models level
fit <- HoltWinters(myts, beta=FALSE, gamma=FALSE)
# double exponential - models level and trend
fit <- HoltWinters(myts, gamma=FALSE)
# triple exponential - models level, trend, and seasonal components
fit <- HoltWinters(myts)

\begin{framed}
	\begin{verbatim}
# predictive accuracy
library(forecast)
accuracy(fit)

# predict next three future values
library(forecast)
forecast(fit, 3)
plot(forecast(fit, 3))
\end{verbatim}
\end{framed}
%======================================================================================== %
\subsection*{ARIMA Models}
The arima() function can be used to fit an autoregressive integrated moving averages model. Other useful functions include:
lag(ts, k)	lagged version of time series, shifted back k observations
diff(ts, differences=d)	difference the time series d times
ndiffs(ts)	Number of differences required to achieve stationarity (from the forecast package)
acf(ts)	autocorrelation function
pacf(ts)	partial autocorrelation function
adf.test(ts)	Augemented Dickey-Fuller test. Rejecting the null hypothesis suggests that a time series is stationary (from the tseries package)
Box.test(x, type="Ljung-Box")	Pormanteau test that observations in vector or time series x are independent
Note that the forecast package has somewhat nicer versions of acf() and pacf() called Acf() and Pacf() respectively.
# fit an ARIMA model of order P, D, Q
fit <- arima(myts, order=c(p, d, q)

\begin{framed}
\begin{verbatim}
# predictive accuracy
library(forecast)
accuracy(fit)

# predict next 5 observations
library(forecast)
forecast(fit, 5)
plot(forecast(fit, 5))
\end{verbatim}
\end{framed}
%======================================================================================================== %
\subsection*{Automated Forecasting}
The forecast package provides functions for the automatic selection of exponential and ARIMA models. The \texttt{ets()} function supports both additive and multiplicative models. The auto.arima() function can handle both seasonal and nonseasonal ARIMA models. Models are chosen to maximize one of several fit criteria.
\begin{framed}
	\begin{verbatim}
	library(forecast)
# Automated forecasting using an exponential model
fit <- ets(myts)

# Automated forecasting using an ARIMA model
fit <- auto.arima(myts)
\end{verbatim}
\end{framed}
%========================================================================================================= %
\end{document}
