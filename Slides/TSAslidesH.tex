
%===================================================%
\begin{frame}
\frametitle{Time Series Analysis}
\textbf{PART H  EXPONENTIAL SMOOTHING AS A FORECASTING METHOD}
Exponential smoothing is a method of forecasting that is also based on using a moving average, but it is a
weighted moving average rather than one in which the preceding data values are equally weighted. The basis for
the weights is exponential because the greatest weight is given to the data value for the period immediately
%CHAP. 16] TIME SERIES ANALYSIS AND BUSINESS FORECASTING 301
preceding the forecast period and the weights decrease exponentially for the data values of earlier periods.
\end{frame}
%===================================================%
\begin{frame}
\frametitle{Time Series Analysis}

There are, in fact, several types of exponential smoothing models, as described in specialized books in business
forecasting. The method presented here is called simple exponential smoothing.
The following algebraic model serves to represent how the exponentially decreasing weights are
determined. Specifically, where a is a smoothing constant discussed below, the most recent value of the time
series is weighted by a, the next most recent value is weighted by a(1 2 a), the next value by a(1 2 a)2, and
so forth, and all the weighted values are then summed to determine the forecast:
^Y
Yt1 ¼ aYt þ a (1  a)Yt1 þ a (1  a)2Yt2 þ " " " þ a (1  a)kYtk (16:17)
where ^YYtþ1 ¼ forecast for the next period
a ¼ smoothing constant (0 # a # 1)
Yt ¼ actual value for the most recent period
Yt1 ¼ actual value for the period preceding the most recent period
Ytk ¼ actual value for k periods preceding the most recent period
\end{frame}
%===================================================%
\begin{frame}
\frametitle{Time Series Analysis}

Although the above formula serves to present the rationale of exponential smoothing, its use is quite
cumbersome. A simplified procedure that requires an initial “seed” forecast but does not require the
determination of weights is generally used instead. The formula for determining the forecast by the simplified
method of exponential smoothing is
^Y
Ytþ1 ¼ ^YYt þ a (Yt  ^YYt) (16:18)
where ^YYtþ1 ¼ forecast for the next period
^Y
Yt ¼ forecast for the most recent period
a ¼ smoothing constant (0 # a # 1)
Yt ¼ actual value for the most recent period
\end{frame}
%===================================================%
\begin{frame}
\frametitle{Time Series Analysis}
Because the most recent time series value must be available to determine a forecast for the following
period, simple exponential smoothing can be used only to forecast the value for the next period in the time
series, not for several periods into the future. The closer the value of the smoothing constant is set to 1.0, the
more heavily is the forecast weighted by the most recent results. See Problem 16.9
\end{frame}
%===================================================%
\begin{frame}
\frametitle{Time Series Analysis}
\textbf{PAART J OTHER FORECASTING METHODS THAT INCORPORATE SMOOTHING}
\begin{itemize}
\item Whereas the moving average is appropriate as the basis for forecasting only when the irregular influence
causes the time series values to vary, simple exponential smoothing is most appropriate only when the cyclical
and irregular influences comprise the main effects on the observed values.
\item  In both methods, a forecast can be
obtained only for the next period in the time series, and not for periods farther into the future. 
\item Other
more complex methods of smoothing incorporate more influences and permit forecasting for several periods
into the future. These methods are briefly described below.
\item Full explanations and descriptions of these methods
are included in specialized textbooks on forecasting.
\end{itemize}
\end{frame}
%===================================================%
\begin{frame}
\frametitle{Time Series Analysis}

Linear exponential smoothing utilizes a linear trend equation based on the time series data. However,
unlike the simple trend equation presented in Section 16.2, the values in the series are exponentially weighted
based on the use of a smoothing constant. As in simple exponential smoothing, the constant can vary from
0 to 1.0.
\end{frame}
%===================================================%
\begin{frame}
\frametitle{Time Series Analysis}
Holt’s exponential smoothing utilizes a linear trend equation based on using two smoothing constants: one
to estimate the current level of the time series values and the other to estimate the slope.
\end{frame}
%===================================================%
\begin{frame}
\frametitle{Time Series Analysis}
Winter’s exponential smoothing incorporates seasonal influences in the forecast. Three smoothing
constants are used: one to estimate the current level of the time series values, the second to estimate the slope of
the trend line, and the third to estimate the seasonal factor to be used as a multiplier.
\end{frame}
%===================================================%
\begin{frame}
\frametitle{Time Series Analysis}
Autoregressive integrated moving average (ARIMA) models are a category of forecasting methods in which
previously observed values in the time series are used as independent variables in regression models. The most
widely used method in this category was developed by Box and Jenkins, and is generally called the Box-Jenkins
method. These methods make explicit use of the existence of autocorrelation in the time series, which is the
correlation between a variable, lagged one or more periods, with itself. As described in Section 15.8, the
\end{frame}
%===================================================%
\begin{frame}
\frametitle{Time Series Analysis}
Durbin-Watson test serves to detect the existence of autocorrelated residuals (serial correlation) in time series
values. A value of the test statistic close to 2.0 supports the null hypothesis that no autocorrelation exists in the
time series. A value below 1.4 generally is indicative of strong positive serial correlation, while a value greater
than 2.6 indicates the existence of strong negative serial correlation.
\end{frame}
%===================================================%
\begin{frame}
\frametitle{Time Series Analysis}

16.11 USING COMPUTER SOFTWARE
Specialized software is available in time series analysis and business forecasting, with much of it
incorporating the more sophisticated techniques that are described in the preceding section of this chapter.
Problems 16.10 through 16.13 illustrate the use of Execustat for determining a linear trend equation, doing
seasonal analysis, using the method of moving averages for the purpose of forecasting, and using simple
exponential smoothing for the purpose of forecasting.
Solved Problems
\end{frame}
%===================================================%
\begin{frame}
\frametitle{Time Series Analysis}

TREND ANALYSIS
16.1 Table 16.1 presents sales data for an 11-year period for a graphics software company (fictional)
incorporated in 1990, as described in Example 1, and for which the time series data are portrayed by the
line graph in Fig. 16-1. Included also are worktable calculations needed to determine the equation for
the trend line. (a) Determine the linear trend equation for these data by the least-square method, coding
1990 as 0 and carrying all values to two places beyond the decimal point. Using this equation determine
the forecast of sales for the year 2001. (b) Enter the trend line on the line graph in Fig. 16.1.
ðaÞ YT ¼ b0 þ b1X
where XX ¼
SX
n
¼
55
11
¼ 5:00
Y
Y ¼
SY
n
¼
12:9
11
¼ 1:17
b1 ¼
SXY # n XX YY
SX 2 # nXX
2 ¼
85:2 # 11(5:00)(1:17)
385 # 11(5:00)2
¼
20:85
110
¼ 0:19
b0 ¼ YY # b1
X
X ¼ 1:17 # 0:19(5:00) ¼ 0:22
YT ¼ 0:22 þ 0:19X(with X ¼ 0 at 1990)
YT (2001) ¼ 0:22 þ 0:19(11) ¼ 2:31 (in millions)
CHAP. 16] TIME SERIES ANALYSIS AND BUSINESS FORECASTING 303
This equation can be used as a beginning point for forecasting, as described in Section 16.6. The slope of 0.19
indicates that during the 11-year existence of the company there has been an average increase in sales of 0.19
million dollars ($190,000) annually.
(b) Figure 16-3 repeats the line graph in Fig. 16-1, but with the trend line entered in the graph. The peak and
trough in the time series that were briefly discussed in Example 1 are now more clearly visible.
\end{frame}
%===================================================%
\begin{frame}
\frametitle{Time Series Analysis}
ANALYSIS OF CYCLICAL VARIATIONS
16.2 Determine the cyclical component for each of the time series values reported in Table 16.1, utilizing the
trend equation determined in Problem 16.1.
Table 16.2 presents the determination of the cyclic relatives. As indicated in the last column of the table, each
cyclical relative is determined by multiplying the observed time series value by 100 and dividing by the trend value.
Thus, the cyclical relative of 90.0 for 1990 was determined by calculating 100(0.20)/0.22.
Table 16.1 Historical Sales for a Graphics Software
Firm, with Worktable to Determine the
Equation for the Trend Line
Year
Coded year
(X)
Sales, in
$millions (Y) X Y X 2
1990 0 $0.2 0 0
1991 1 0.4 0.4 1
1992 2 0.5 1.0 4
1993 3 0.9 2.7 9
1994 4 1.1 4.4 16
1995 5 1.5 7.5 25
1996 6 1.3 7.8 36
1997 7 1.1 7.7 49
1998 8 1.7 13.6 64
1999 9 1.9 17.1 81
2000 10 2.3 23.0 100
Total 55 $12.9 85.2 385
Fig. 16-3
% 304 TIME SERIES ANALYSIS AND BUSINESS FORECASTING [CHAP. 16
\end{frame}
%===================================================%
\begin{frame}
\frametitle{Time Series Analysis}
16.3 Construct a cycle chart for the sales data reported in Table 16.1, based on the cyclical relatives
determined in Table 16.2.
The cycle chart is presented in Fig. 16-4, and includes the peak and trough observed previously in Fig. 16-3.
\end{frame}
%===================================================%
\begin{frame}
\frametitle{Time Series Analysis}
MEASUREMENT OF SEASONAL VARIATIONS
16.4 Table 16.3 presents quarterly sales data for the graphics software company for which annual
data are reported in Table 16.1. Determine the seasonal indexes by the ratio-to-moving-average
method.
Table 16.4 is concerned with the first step in the ratio-to-moving-average method, that of computing the ratio
of each quarterly value to the 4-quarter moving average centered at that quarter.
The 4-quarter moving totals are centered between the quarters in the table because as a moving total of an even
number of quarters, the total would always fall between two quarters. For example, the first listed total of 1,500 (in
$1,000s) is the total sales amount for the first through the fourth quarters of 1995. Since four quarters are involved,
the total is centered vertically in the table between the second and third quarter.
Table 16.2 Determination of Cyclical Relatives
Coded year Sales, in $millions Cyclical relative
Year (X) Actual (Y) Expected (Y) 100 Y/Y
1990 0 0.20 0.22 90.9
1991 1 0.40 0.41 97.6
1992 2 0.50 0.60 83.3
1993 3 0.90 0.79 113.9
1994 4 1.10 0.98 112.2
1995 5 1.50 1.17 128.2
1996 6 1.30 1.36 95.6
1997 7 1.10 1.55 71.0
1998 8 1.70 1.74 97.7
1999 9 1.90 1.93 98.4
2000 10 2.30 2.12 108.5






FORECASTING BASED ON TREND AND SEASONAL FACTORS
16.6 Step down the trend equation developed in Problem 16.1 so that it is expressed in terms of quarters
rather than years. Also, adjust the equation so that the trend values are in thousands of dollars instead of
millions of dollars. Carry the final values to the first place beyond the decimal point.
\[ Y_T (quarterly) =  \frac{b_0}{4} -1.5 \left( \frac{b_1}{16} \right) + \left( \frac{b_1}{16} \right)X\]

\[  Y_T (quarterly) =  \frac{0.22}{4} -1.5 \left( \frac{0.19}{16} \right) + \left( \frac{0.19}{16} \right)X\]

\[ Y_T (quarterly) = 0.0550   0.0178 + 0.0119X
\[ Y_T (quarterly) = 0.0372 + 0.0119X\]


