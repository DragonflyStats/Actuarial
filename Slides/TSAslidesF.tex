
%===================================================%
\begin{frame}
\frametitle{Time Series Analysis}
\textbf{PART E: APPLYING SEASONAL ADJUSTMENTS}
\begin{itemize}

\item One frequent application of seasonal indexes is that of adjusting observed time series data by removing the
influence of the seasonal component from the data. 
\item Such adjusted data are called seasonally adjusted data, or
deseasonalized data. 
\item Seasonal adjustments are particularly relevant if we wish to compare data for different
months to determine if an increase (or decrease) relative to seasonal expectations has taken place.
\end{itemize}
\end{frame}
%===================================================%
\begin{frame}
\frametitle{Time Series Analysis}
\textbf{EXAMPLE 3. }
\begin{itemize}
\item An increase in lawn fertilizer sales of 10 percent from April to May of a given year represents a relative
decrease if the seasonal index number for May is 20 percent above the index number for April. 
\item In other words, if an increase
occurs but is not as large as expected based on historical data, then relative to these expectations a relative decline in demand
has occurred.
\end{itemize}
\end{frame}

%===================================================%
\begin{frame}
\frametitle{Time Series Analysis}
\begin{itemize}
\item The observed monthly (or quarterly) time series values are adjusted for seasonal influence by dividing each
value by the monthly (or quarterly) index for that month. 
\item The result is then multiplied by 100 to maintain the
decimal position of the original data.
\item The process of adjusting data for the influence of seasonal variations can
be represented by
\end{itemize}
%% EQUATION

Although the resulting values after the application of (16.11) are in the same measurement units as the
original data, they do not represent actual data. Rather, they are relative values and are meaningful for
comparative purposes only. See Problem 16.5.

\end{frame}
%===================================================%
\begin{frame}
\frametitle{Time Series Analysis}
\textbf{PART F: FORECASTING BASED ON TREND AND SEASONAL FACTORS}
\begin{itemize}
\item A beginning point for long-term forecasting of annual values is provided by use of the trend line (16.2)
equation. 
\item However, a particularly important consideration in long-term forecasting is the cyclical component of
the time series. 
\item There is no standard method by which the cyclical component can be forecast based on
historical time series values alone, but certain economic indicators (see Section 16.7) are useful for anticipating
cyclical turning points.
\end{itemize}

\end{frame}
%===================================================%
\begin{frame}
\frametitle{Time Series Analysis}
\begin{itemize}
\item For short-term forecasting, one possible approach is to deseasonalize the most-recent observed value and
then to multiply this deseasonalized value by the seasonal index for the forecast period. 
\item This approach assumes
that the only difference between the two periods will be the difference that is attributable to the seasonal
influence. 
\item An alternative approach is to use the projected trend value as the basis for the forecast and then adjust
it for the seasonal component. 
\item When the equation for the trend line is based on annual values, one must first
``step down" the equation so that is expressed in terms of months (or quarters). 
\end{itemize}
\end{frame}
\end{document}
%===================================================%
\begin{frame}
\frametitle{Time Series Analysis}A trend equation based on annual
data is modified to obtain projected monthly values as follows:
YT ¼
b0
12
þ
b1
12  € X
12  € ¼
b0
12
þ
b1
144
X (16:12)
A trend equation based on annual data is modified to obtain projected quarterly values as follows:
YT ¼
b0
4
þ
b1
4  € X
4  € ¼
b0
4
þ
b1
16
X (16:13)

\end{frame}
%===================================================%
\begin{frame}
\frametitle{Time Series Analysis}
\begin{itemize}
\item The basis for the above modifications is not obvious if one overlooks the fact that trend values are not
associated with points in time, but rather, with periods of time. 
item Because of this consideration, all three elements
in the equation for annual trend (b0, b1, and X) have to be stepped down.
\item By the transformation for monthly data in (16.12), the base point in the year that was formerly coded X ¼ 0
would be at the middle of the year, or July 1. 
\item Because it is necessary that the base point be at the middle of the
first month of the base year, or January 15, the intercept b0/12 in the modified equation is then reduced by 5.5
times the modified slope. A similar adjustment is made for quarterly data. 
\end{itemize}

\end{frame}
%===================================================%
\begin{frame}
\frametitle{Time Series Analysis}
Thus, a trend equation which is
modified to obtain projected monthly values and with X ¼ 0 placed at January 15 of the base year is
YT ¼
b0
12
" (5:5)
b1
144  € þ
b1
144
X (16:14)
\end{frame}
%===================================================%
\begin{frame}
\frametitle{Time Series Analysis}
Similarly, a trend equation that is modified to obtain projected quarterly values and with X ¼ 0 placed at the
middle of the first quarter of the base year is
YT ¼
b0
4
" (1:5)
bi
16  € þ
b1
16
X (16:15)
Problem 16.6 illustrates the process of stepping down a trend equation. After monthly (or quarterly) trend
values have been determined, each value can be multiplied by the appropriate seasonal index (and divided by
100 to preserve the decimal location in the values) to establish a beginning point for short-term forecasting.
See Problem 16.7.

\end{frame}
