The Granger causality test is a statistical hypothesis test for determining whether one time series is useful in forecasting another. 

Ordinarily, regressions reflect "mere" correlations, but Clive Granger argued that causality in economics could be reflected by measuring the ability of predicting the future values of a time series using past values of another time series. Since the question of "true causality" is deeply philosophical, econometricians assert that the Granger test finds only "predictive causality".
 
A time series X is said to Granger-cause Y if it can be shown, usually through a series of t-tests and F-tests on lagged values of X (and with lagged values of Y also included), that those X values provide statistically significant information about future values of Y.
 
Granger also stressed that some studies using "Granger causality" testing in areas outside Economics reached "ridiculous" conclusions. "Of course, many ridiculous papers appeared", he said in his Nobel Lecture, December 8, 2003. 

However, it remains a popular method for causality analysis in time series due to its computational simplicity. 

The original definition of Granger causality does not account for latent confounding effects and does not capture instantaneous and non-linear causal relationships, though several extensions have been proposed to address these issues.[4]
